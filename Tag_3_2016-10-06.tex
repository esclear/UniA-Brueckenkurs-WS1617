\documentclass[14pt,a4paper]{article}

\usepackage{setspace}
\usepackage[margin=2cm]{geometry}
\usepackage[utf8]{inputenc}
\usepackage[ngerman]{babel}
\usepackage{graphicx}
\usepackage{amsmath}
\usepackage{amsfonts}
\usepackage{hyperref}

\usepackage{lastpage}
\usepackage{fancyhdr}
\pagestyle{fancy}
\fancyhf{}
\renewcommand{\headrulewidth}{0pt}
\fancyfoot[L]{Daniel Albert}
\fancyfoot[C]{Alle Mitschriften ohne Garantie auf Korrektheit oder Vollständigkeit}
\fancyfoot[R]{\thepage{} / \pageref{LastPage}}


\newcommand{\dotcup}{\ensuremath{\mathaccent\cdot\cup}}
\newcommand*\rfrac[2]{{}^{#1}\!/_{#2}}
\newcommand{\N}{\ensuremath{\mathbb{N}}}
\newcommand{\Z}{\ensuremath{\mathbb{Z}}}
\newcommand{\Q}{\ensuremath{\mathbb{Q}}}
\newcommand{\Nzero}{\ensuremath{\N_0}}

\begin{document}
	\begin{center}
		\Huge\textbf{Brückenkurs – Tag 3 – „3. Ausgabe“}
	\end{center}
	\par

	\section{Die natürlichen Zahlen und das Induktionsprinzip}
	\subsection{Beispiel}
		Von Tag 2:
			\paragraph{Satz}
				$$ \Sigma_{k=1}^{n} k = \frac{1}{2} \cdot n \cdot (n + 1) $$
			\paragraph{Folgerung (\textit{Korollar})}
				$$ \Sigma_{k=1}^{n} ( 2 \cdot k -1 ) = n^2 $$
			\paragraph{Beweis}
				$$ \Sigma_{k=1}^{n} (2k-1) = \Sigma_{k=1}^{2n} k - \Sigma_{k=1}^{n} 2k $$
				Mit Formel aus Satz auf die Formel angewendet:
				$$ \frac{1}{2} \cdot 2 \cdot n \cdot (2n+1) -2 \cdot \frac{1}{2} n(n+1) = 2n^2 + n - n^2 -n = n^2 $$
				Es wird zuerst die Summe aller Zahlen von $1$ bis $2n$ addiert, danach die Summe aller geraden Zahlen abgezogen

				Hier auch implizite Verwendung der Assoziativität und Kommutivität der Addition.

	\subsection{Weiteres Beispiel}
		\paragraph{Satz} Sei $ x \neq 1 $. Dann gilt: $\Sigma_{k=0}^{n} x^k = \frac{1-x^{n+1}}{1-x} $ („Geometrische Summe“)
		\paragraph{Beispiel}
			$$ 1 + 2 + 4 + \dots 2^{63} = \frac{1-2^{64}}{1-2} = 2^{64} - 1 = 18.446.744.073.709.551.615 $$
		\paragraph{Beweis 1} Ansatz der vollständigen Induktion:
			\subparagraph{$n=0$}
				$ x^0 = 1 ; \frac{1-x^2}{1-x} = 1 $ Formel stimmt also für $n=0$
			\subparagraph{$ n \implies n+1 $}
				$$ \Sigma_{k=0}^{n+1} x^k = x^{n+1} + \Sigma_{k=0}^{n} x^k = (I.V)  x^{n+1} + \frac{1-x^{n+1}}{1-x} = \frac{x^{n+1}(1-x) + 1 -x^{n+1}}{1-x} = \frac{1-x^{n+2}}{1-x} $$
		\paragraph{Beweis 2}
			$$ \Sigma_{k=0}^n x^k = \frac{1-x^{n+1}}{1-x} \Leftrightarrow (1-x) \Sigma_{k=0}^n x^k = 1-x^{n+1} = \Sigma_{k=0}^{n} x^k - \Sigma_{k=0}^n x^{k+1} = \Sigma_{k=0}^n x^k - \Sigma_{k=1}^{n+1} x^k = x^0 - x^{n+1} = 1 - x^{n+1}  $$
			Für $ x \neq 1 $.
			q.e.d.
	\subsection{Äquivalenz- und Induktionsprinzip}
	\paragraph{Satz}
		Jede nicht-leere Teilmenge von $ \mathbb{N}_0 $ besitzt ein kleinstes Element. („ $\mathbb{N}_0$ ist \textit{wohlgeordnet} “)
	\paragraph{Beweis}
		Sei $ M \subseteq \mathbb{N}_0$ ohne kleinstes Element. Wir wollen zeigen dass: $ M = \emptyset $, d.h. $ P = \{ n \in \mathbb{N}_0 | 0, 1, \dots, n \notin M \} = \mathbb{N}_0 $  % $P = \mathbb{N}_0 \setminus M = \mathbb{N}_0 $
		Hierbei Anwendung des \textit{Peano-Axioms}:
		\subparagraph{$0 \in P$}
			Wäre $0 \notin P$, so wäre $0 \in M$, insbesondere kleinstes Element von $M$. Dies ist ein Widerspruch, also $0 \in P$.
		\subparagraph{$ n \in P \implies n+1 \in P$}
			Wäre $n+1 \notin P$. Dann wäre eine der Zahlen $ 0, \dots, n+1 \in M$.
			Da aber nach Voraussetzung $n \in P$, ist $0, \dots, n \notin M$. Also $n+1 \in M$.
			Insbesondere ist $n+1$ kleinstes Element. Widerspruch, also ist $n+1 \in P$.

	\section{Die ganzen und die rationalen Zahlen}
		\subsection{Relation}
			Eine \textbf{Relation} auf einer Menge $M$ ist eine Teilmenge $ R \subseteq M \times M $
			Wir schreiben $ x \sim y :\Leftrightarrow. (x,y) \in R$ für $x,y \in M$.

			\paragraph{Beispiel} $ x \leq y $ auf $ \mathbb{N}_0$:

				[Skizze: Punkte auf Gitter, $x,y \leq 4 \in \mathbb{N}_0$. Oberhalb und auf der Diagonale blaue Menge.]

				\begin{tabular}{ c | c | c | c | c   c }
					  & 0 & 1 & 2 & 3 & y \\
					0 & x &   &   &   &   \\
					1 & x & x & x & x &   \\
					2 & x &   & x &   &   \\
					3 & x &   &   & x &   \\
					x &   &   &   &   &   \\
				\end{tabular}
			\paragraph{Definition}
				Eine Relation auf $M$ heißt \textbf{Äquivalenzrelation}, falls sie:
				\begin{enumerate}
					\item \textbf{reflexiv} ist, d.h. $ x \sim x $ für alle $ x \in M$.
					\item \textbf{symmetrisch} ist, d.h. $x \sim y \implies y \sim x $ für alle $ x,y \in M $.
					\item \textbf{transitiv} ist, d.h. $ x \sim y \land y \sim z \implies x \sim z$ für alle $x,y,z \in M$.
				\end{enumerate}
			\paragraph{Beispiel}
				Die Gleichheitsrelation auf einer Menge ist eine Äquivalenzrelation
			\paragraph{Beispiel}
				Sei $M$ eine Menge von Menschen. Die Relation „ist verwand mit“ (im Sinne von „gehört zur gleichen Familie“) ist eine Äquivalenzrelation.
			\paragraph{Beispiel}
				Sei $M$ eine Menge von Menschen. Die Relation „hat im gleichen Monat Geburtstag“ ist eine Äquivalenzrelation.

				Dabei ist $ M = M_1 \dotcup M_2 \dotcup \dots \dotcup M_{12} $. Die $M_1$ heißen die \textbf{Äquivalenzklassen} der Relation und stehen hier für die Monate.

			\paragraph{Beispiel}
				Relation $ \sim $ auf $ Z $ mit $ x \sim y :\Leftrightarrow x-y $ gerade.
				Ist reflexiv und symmetrisch.
				ist transitiv? $ x \sim y, y \sim z \implies x-y$ gerade, $ y-z$ gerade. $\implies (x-y) + (y-z) = x-z$ gerade $\implies x \sim z $
				Ist also Äquivalenzrelation.

				\subparagraph{Äquivalenzklassen}
					In diesem Beispiel: $ Z = \{Gerade Zahlen\} \dotcup \{Ungerade Zahlen\} $


			\paragraph{Definition} Sei $ \sim $ eine Relation auf einer Menge $M$. Für $ x \in M $ heißt dann $ [x]_{(\sim)} := \{ y \in M | x \sim y \}$ die \textbf{Äquivalenzklasse} zu $x$.
				\subparagraph{Beispiel}
					$ [Peter]_{verwandt} = Peters Familie $
			\paragraph{Satz}
				Es gilt für alle Äquivalenzrelationen auf eine Menge $M$ mit $x,y \in M$:
				\begin{enumerate}
					\item $ x \in [x] $
					\item $ x \sim y \implies [x] = [y] $
					\item $ [x] \neq [y] \implies [x] \cap [y] = \emptyset $
				\end{enumerate}
			\paragraph{Beweis}
				\begin{enumerate}
					\item $ x \in [x] \Leftrightarrow x \sim x $ ok
					\item Sei $x \sim y $ Zu \textbf{zeigen}: $ [x] = [y] $. \\
						$ z \in [x] \Leftrightarrow x \sim z \implies^{x \sim y}_{y \sim x} y \sim z \Leftrightarrow z \in [y] $
					\item Wir zeigen: $ [x] \cap [y] \neq \emptyset \implies [x] = [y] $ \\
						Es existiert also $ z \in [x] \cap [y]$, d.h. $ z \in [x] $ und $ z \in [y] $, d.h. $ x \sim z$, $y \sim z \implies x \sim y \implies x] = [y]$.
				\end{enumerate}
				q.e.d.

			\paragraph{Definition}
				$x$ heißt \textbf{Repräsentant} seiner Äquivalenzklasse $ [x] $: \\
				$ M = \dotcup [x]$. \{$x$ Repräsentantensystem\}

			\paragraph{Definition}
				Sei $R$ eine Äquivalenzrelation auf einer Menge $M$. Dann heißt $ \rfrac{M}{R} := \{ [x]_R | x \in R \} $ der \textbf{Quotioent von M nach R}.
		\subsection{Konstruktion der ganzen Zahlen}
			Erklärung ganzer Zahlen als Paar zweier natürlicher Zahlen. Dabei Subtraktion der Zahlen.
			Beispiel: Kontostand zusammengesetzt aus Einzahlungen und Abhebungen.

			$ ( Einzahlungen, Abhebungen) \sim (Einzahlungen', Abhebungen') \Leftrightarrow E + A' = E' + A $
			Auf der Menge der Paare $(n, m)$ natürlicher Zahlen definieren wir die Relation $(n,m) \sim (a,b) :\Leftrightarrow n +b = m + a $ \\
			Es ist $ \sim $ eine Äquivalenzrelation: Ist reflexiv und symmetrisch.
			Transitivität: $$ (n,m) \sim (a,b) \land (a,b) \sim (u,v) \implies n+b = m+a \land a + v = b + u \implies u + b + a + v = m + a + b + u \implies n + v = m + u \implies (n, m) \sim (u, v) $$

			Die Äquivalenzklasse zum Paar $(n,m)$ heißt $[n,m]$

			\paragraph{Beispiel} $ [3,2] \sim [5,4] $

			\paragraph{Definition}
				$$ Z = \rfrac{\mathbb{N}_0 \times \mathbb{N}_0}{\sim} = \{ [n,m] \; | \; n,m \in \mathbb{N}_0 \} $$
				Jeder natürlichen Zahl $n$ entspricht eine ganze Zahl $ [n, 0] $.
				$ \to \mathbb{N}_0 \subseteq Z $ \\ $n \mapsto [n,0]$.

			\subparagraph{Negative Zahlen}
				$ - [n,m] = [m,n] $
			\subparagraph{Beispiel}
				$ n \in \mathbb{N}_0 \; ; \; -n = -[n,0]  = [0,n]$
				Ist diese Relation wohldefiniert?
				$ - [7,2] = [2,7] $
				\subparagraph{Zu zeigen} $[n,m] \sim [a,b] \implies [m,n] \sim [b,a]$ \\
					Begründung: Wenn $ [n,m] \sim [a,b] \Leftrightarrow n + b = m + a \Leftrightarrow m+a = n+b \Leftrightarrow [m,n] \sim [b,a] $
			\subparagraph{Addition}
				$ [n,m] + [a,b] := [n+a, m+b] $
			\subparagraph{Multiplikation}
				$ [m,n] \cdot [a,b] := [ma+nb, \; na+mb] $
		\subsection{Rationale Zahlen}
			Auf der Menge $ Z \times N_{>0} $ betrachten wir die Relation $(a,s) \sim (b,t) \Leftrightarrow a \cdot t = b \cdot s $
			\paragraph{Rechnung} $\sim$ ist Äquivalenzrelation.
			Die Äquivalenzklasse zu $(a,s)$ bezeichnen wir mit $\frac{a}{s}$. \\
			$ \mathbb{Q} := \rfrac{\mathbb{Z} \times \mathbb{N}_0}{N}$
			\paragraph{Addition}
				$$ \frac{a}{s} + \frac{b}{t} := \frac{at + bs}{st} $$

				$$ \frac{b'}{t'} = \frac{b}{t} \Leftrightarrow t b' = t' b \implies \frac{at + bs}{st} = \frac{at' + b's}{s t'} \Leftrightarrow t'b = t b' $$

		\subsection{Binomialkoeffizienten}
			Sei $x$ eine (reelle) Zahl, $ k \geq 0 $ natürliche Zahl.
			Dann heißt $ \left(\!\begin{array}{c} x \\ k \end{array}\!\right) := \frac{x \cdot (x-1) \cdot \dots \cdot (x - k + 1)}{k!} $ der \textbf{Binomialkoeffizient} „x über k“.
			\paragraph{Spezialfall}
				Sei $ 0 \leq k \leq n $ eine natürliiche Zahl. Dann ist $ \left(\!\begin{array}{c} n \\ k \end{array}\!\right) = \frac{n!}{k!(n-k)!} $
\end{document}
