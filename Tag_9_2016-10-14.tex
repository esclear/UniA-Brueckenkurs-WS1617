\documentclass[14pt,a4paper]{article}

\usepackage{setspace}
\usepackage[margin=2cm]{geometry}
\usepackage[utf8]{inputenc}
\usepackage[ngerman]{babel}
\usepackage{graphicx}
\usepackage{amsmath}
\usepackage{amsfonts}
\usepackage{hyperref}

\usepackage{mathrsfs}
\usepackage{amssymb}

% Footer

\usepackage{lastpage}
\usepackage{fancyhdr}
\pagestyle{fancy}%
\fancyhf{}
\renewcommand{\headrulewidth}{0pt}%
\fancyfoot[L]{Daniel Albert}
\fancyfoot[C]{Alle Mitschriften ohne Garantie auf Korrektheit oder Vollständigkeit}
\fancyfoot[R]{\thepage{} / \pageref{LastPage}}

%Eigene Makros

\newcommand{\dotcup}{\ensuremath{\mathaccent\cdot\cup}}
\newcommand*\rfrac[2]{{}^{#1}\!/_{#2}}
\newcommand{\N}{\ensuremath{\mathbb{N}}}
\newcommand{\Z}{\ensuremath{\mathbb{Z}}}
\newcommand{\Q}{\ensuremath{\mathbb{Q}}}
\newcommand{\C}{\ensuremath{\mathbb{C}}}
\newcommand{\R}{\ensuremath{\mathbb{R}}}
\newcommand{\Nzero}{\ensuremath{\N_0}}

\newcommand{\iu}{{i\mkern1mu}}

\begin{document}
	\begin{center}
		\Huge\textbf{Brückenkurs – Tag 9 – 2016-10-14}
	\end{center}
	%\par

  \setcounter{section}{12}
  \section{Auswahlaxiom, Zornsches Lemma und Ultrafilter}

  \subsection{Auswahlaxiom}

  \paragraph{Definition}

  Ist $M$ eine Menge nicht-leerer Mengen, so existiert dazu eine Auswahlmenge,
  das heißt: eine Menge $X$, so dass $\forall U \in M \exists!\, a \in U$. mit $a
  \in X$\footnote{$\exists\text{!}$ bedeutet: „es existiert genau ein Element“}.

  Sei $Z$ eine Menge, $\mathscr{X} \subseteq P(Z)$, also ist $\mathscr{X}$
  eine Menge von Teilmengen von $Z$.

  \subparagraph{Definition}
  Eine \textbf{Kette in $\mathscr{X}$} ist eine Teilmenge $ \mathscr{Y}
  \subseteq \mathscr{X}$ mit $ \forall Y_1, Y_2 \in \mathscr{Y} : Y_1 \subseteq
  Y_2 \land Y_2 \subseteq Y_1 $

  \subsection{Zornsches Lemma}
  Sei $Z, \mathscr{X}$ wie eben.
  Zusätzlich gelte:
  \begin{enumerate}
    \item Ist $X' \subseteq X \in \mathscr{X}$, so auch $X' \in \mathscr{X}$
    \item Ist $\mathscr{Y \subseteq X}$ eine Kette, so ist $\cup \mathscr{Y} =
      \cup Y \in \mathscr{X}$.
  \end{enumerate}

  Dann besitzt $\mathscr{X}$ ein \textbf{maximales Element} $X_0 \in \mathscr{X}$ bzgl.
  „$\subseteq$“, d.h. $\forall X \in \mathscr{X}: X \supseteq X_0 \implies X =
  X_0$

  \paragraph{Beweisidee}
  \begin{itemize}
  \item Wegen 2. (Wähle $\mathscr{Y = \emptyset \subseteq X}$ (Kette)) ist $\emptyset
  = \cup \emptyset \in \mathscr{X}$.
  \item Falls $\emptyset$ maximal in $\mathscr{X}$, sind wir fertig.
  \item Ansonsten gibt es $X_1 \in \mathscr{X}$ mit $X_0 \subsetneqq X_1$.
  \item Entweder ist $X_1$ maximal oder wir machen weiter … \\
    $X_0 \subsetneqq X_1 \subsetneqq X_2 \subsetneqq X_3 \subsetneqq \ldots
    \subsetneqq X_\omega$
  \end{itemize}

  Breche der Prozess nicht ab (ansonsten wären wir nach $n \in \Nzero$ Schritten
  fertig.)\\
  Wegen 2. ist $X_\omega = \cup_{i=0}^{\infty} X_i \in \mathscr{X}$.

  Ist $X_\omega$ immer noch nicht maximal, so finden wir $X_\omega \subsetneqq
  X_{\omega + 1} \subsetneqq \ldots \subsetneqq X_{\omega + n} \subsetneqq
  \ldots$

  Bricht dies immer noch nicht ab, so ist $X_{\omega \cdot 2} = \cup_{n=0}^{\infty} X_{\omega +
    n}$ der nächste Kandidat.\\%
  %
  $X_0 \subsetneqq X_1 \subsetneqq X_2 \subsetneqq \ldots X_\omega$\\
  $X_\omega  \subsetneqq X_{\omega+1} \subsetneqq X_{\omega+2} \subsetneqq
  X_{\omega+3} \subsetneqq \ldots \subsetneqq X_{\omega \cdot 2}$\\
  $X_{\omega\cdot2}  \subsetneqq X_{\omega\cdot2+1} \subsetneqq X_{\omega\cdot2+2} \subsetneqq
  X_{\omega\cdot2+3} \subsetneqq \ldots \subsetneqq X_{\omega \cdot 3}$

  \paragraph{Korollar}
  Sei $(Z, \leq)$ eine \textbf{teilweise geordnete} Menge, das heißt es gilt:
  \begin{enumerate}
    \item $\forall z \in Z : z \leq z$.
    \item $\forall x,y \in Z : x \leq y \land y \leq x \implies x = y $
    \item $\forall x,y,z \in Z : x \leq y \land y \leq z \implies x \leq z$
  \end{enumerate}

  Besitzt dann jede \textbf{Kette} $Y$ in $Z$ (d.h.\ jede vollständig geordnete Teilmenge
  von $Y \subseteq Z$) eine \textbf{obere Schranke} in $Z$, das heißt $\exists z
  \in Z \forall y \in Y : y \leq z$, dann besitzt $Z$ ein maximales Element $z_0
  \in Z$, das heißt $\forall z \in Z : z \geq z_0 \implies z = z_0$.

  \subparagraph{Beweis}
  Sei $\mathscr{X} \subseteq P(Z)$ die Menge der Ketten von $(Z, \leq)$.
  Dann sind 1. und 2. vom Zornschen Lemma erfüllt.
  Damit existiert eine maximale Kette $X_0 \in \mathscr{X}$.

  Nach Voraussetzung des Korollars besitzt $X_0$ eine obere Schranke $z_0 \in
  Z$.

  \subparagraph{Annahme}
  $z_0$ ist nicht maximal, das heißt es existiert  $z_1 \in Z$ mit $z_1 \geq
  z_0, z_1 \neq z_0$.
  Dann wäre aber $X_0 \cup \{z_1\}$ eine echt größere Kette als $X_0$.
  Dies wäre aber ein \textbf{Widerspruch} zur Maximalität von $X_0$.

  \subsection{Ultrafilter}
  \paragraph{Definition}
  Sei $X$ eine Menge. Ein \textbf{Filter} $F$ auf $X$ ist eine Teilmenge $F
  \subseteq P(X)$ mit
  \begin{enumerate}
  \item $X \in F$
  \item $\emptyset \notin F$
  \item $\forall A \in F : B \supseteq A \Rightarrow B \in F$
  \item $\forall A,B \in F \Rightarrow A \cap B \in F$
  \end{enumerate}

  \subparagraph{Beispiel}
  Sei $x_0 \in X$ ein Element einer Menge.
  Dann ist $F := \{ A \subseteq X | x_0 \in A\}$ ein Filter, der von $x_0$ erzeugte Filter.

  Filter, die nicht von einem Element erzeugt werden, heißen \textbf{frei}.

	\subparagraph{Beispiel}
		Sei $S$ eine unendlich große Menge.
		Dann ist $F := \{A \subseteq X | X \setminus A \text{ endlich} \}$ ein Filter, der sogenannte \textbf{Fréchet-Filter} auf $X$.
		

	\paragraph{Definition}
		Ein \textbf{Ultrafilter} auf $X$ ist ein Filter mit 5. $\forall A \subseteq X : A \in F \lor X \setminus A \in F$.
		
		\subparagraph{Beispiel}
			Nicht freie Filter\footnote{Bspw. Fréchet-Filter} sind Ultrafilter.
		\subparagraph{Frage}
			Gibt es freie Ultrafilter?
		\subparagraph{Satz}
			Ist $F$ ein Filter auf $X$, so gibt es einen Ultrafilter $\hat{X}$ auf $X$ mit $F \subseteq \hat{F}$.
		\subparagraph{Folgerung}
			Auf jeder unendlichen Menge gibt es einen freien Ultrafilter.
		\paragraph{Beweis (Folgerung)}
			Wähle einen Ultrafilter, der den Fréchet-Filter umfasst.
			$\square$
		\paragraph{Beweis (Satz)}
			Sei $Z$ die Menge der Filter $\widetilde{F}$ mit $\widetilde{F} \supseteq F$.
			Es ist $Z$ bezüglich „$\subseteq$“ teilweise geordnet.
			Jede Kette $\mathbb{F}$ in $Z$, also jede Kette von Filtern besitzt eine obere Schranke in $Z$, nämlich $\cup_{\widetilde{F} \in F} \widetilde{F} $.

			Zu überprüfen, dass dies ein Filter ist, also in $Z$ liegt.\
			
			z.B. Filgereigenschaft 4. : $A,B \in \cup_{\widetilde{F} \in \mathbb{F}} \widetilde{F} \stackrel{?}{\Rightarrow} A \cap B \in \cup_{\tilde{F} \in \mathbb{F}} \mathbb{F}$

			$\rightsquigarrow$ Da $\mathbb{F}$ Kette,  $\tilde{F}_1 \subseteq \tilde{F}_2$ oder $\tilde{F}_2 \subseteq \tilde{F}_1$.
			Ohne Beschränkung der Allgemeinheit: $\tilde{F}_1 \subseteq \tilde{F}_2$.

			Also $A,B \in \tilde{F}_2 \Rightarrow A \cap B \in \tilde{F}_2 \in \mathbb{F} \Rightarrow A \cap B \in \cup \mathbb{F}$.

			Nach Zorn besitzt $Z$ ein maximales Element $\hat{F}$.

			\textbf{Behauptung}: $\hat{F}$ ist Ultrafilter.

			\textbf{Begründung}
			Unter der Annahme, dass $\hat{F}$ ein Ultrafilter ist, gibt es ein $A \subseteq X$ mit $A \notin \hat{F}$ und $X \setminus A \notin \hat{F}$.
			
			\textbf{Definition}:
			$\mathscr{Y} := \{ G \subseteq X | \exists F \in \hat{F} : G \supseteq F \cap A\}$

			Damit ist $\mathscr{G}$ ein Filter; wegen $A \in \mathscr{G}$, aber $A \notin \hat{F}$ ist $\hat{F} \neq \mathscr{G}$.
			Aber $\hat{F} \subseteq \mathscr{G}$.

			Damit $\hat{F}$ nicht maximal. \textbf{Widerspruch!}
			$\square$

\end{document}
