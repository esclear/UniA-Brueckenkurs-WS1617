\documentclass[14pt,a4paper]{article}

\usepackage{setspace}
\usepackage[margin=2cm]{geometry}
\usepackage[utf8]{inputenc}
\usepackage[ngerman]{babel}
\usepackage{graphicx}
\usepackage{amsmath}
\usepackage{amsfonts}
\usepackage{hyperref}

\usepackage{lastpage}
\usepackage{fancyhdr}
\pagestyle{fancy}
\fancyhf{}
\renewcommand{\headrulewidth}{0pt}
\fancyfoot[L]{Daniel Albert}
\fancyfoot[C]{Alle Mitschriften ohne Garantie auf Korrektheit oder Vollständigkeit}
\fancyfoot[R]{\thepage{} / \pageref{LastPage}}


\newcommand{\dotcup}{\ensuremath{\mathaccent\cdot\cup}}
\newcommand*\rfrac[2]{{}^{#1}\!/_{#2}}
\newcommand{\N}{\ensuremath{\mathbb{N}}}
\newcommand{\Z}{\ensuremath{\mathbb{Z}}}
\newcommand{\Q}{\ensuremath{\mathbb{Q}}}
\newcommand{\Nzero}{\ensuremath{\N_0}}

\begin{document}
	\begin{center}
		\Huge\textbf{Brückenkurs – Tag 5 – 2016-10-10}
	\end{center}
	\par

  \section*{In der letzten Ausgabe}
  In der letzten Vorlesung behandelt: Primfaktorzerlegung.
  Weiterhin ist unbekannt, wie viele Primzahlen existieren. Ist ihre Zahl
  unbeschränkt?

  \setcounter{section}{5}
  \section{Primzahlen}
  \paragraph{Satz (Euklid)}
  Es gibt undendlich viele Primzahlen.

  \subparagraph{Beweis}
  Seien $ p_0, \dots p_{n-1} $ Primzahlen.

  Dann können wir eine Primzahl $p_n$
  konstruieren mit $p_n \notin \{p_0, \dots, p_{n-1}\}$:
  Dazu betrachte: $ e := p_0 \dots p_{n-1} + 1 = q_1 \dots q_s $ mit
  Primzahlen $ q_1, \dots, q_s$ (PFZ)

%  Dabei Überlegung: $q_s$ teilt linke Seite, aber kein $p$ teilt die rechte
%  Seite (wg. der Addition $+1$ führt zu Rest 1).

  Da die $P-i$ jeweils $e$ nicht teilen (Rest $1$!), die $q_j$ aber $e$ teilen,
  sind die $q_j$ von $p_i$ verschieden.
  Damit ist $p_n := q_i$ die gesuchte Primzahl.

  \subparagraph{Beispiel}
  $$
  \begin{array}{r c c c l}
    \emptyset & \rightsquigarrow & 1 + 1 & = & 2 = 2^1  \\
  {2} & \rightsquigarrow & 2 + 1 & = & 3 = 3^1 \\
  {2,3} & \rightsquigarrow & 2 \cdot 3 + 1 & = & 7 = 7^1 \\
  {2,3,7} & \rightsquigarrow & 2 \cdot 3 \cdot 7 + 1 & = & 43 = 43^1 \\
  {2, 3,7,43} & \rightsquigarrow & 2 \cdot 3 \cdot 7 \cdot 43 + 1 & = & 1807 = 13 \cdot
    139 \\
  \end{array}
  $$

  \paragraph{Primzahlsatz}
  Sei $\pi(x)$ die Anzahl der Primzahlen $ \le x$.
  Dann gilt: $\pi(x) \approx \frac{x}{\ln x}$, d.h.

  $$ \lim_{x \to \infty} \rfrac{\pi(x)}{\rfrac{x}{\log x}} = 1$$

  $$ \pi(1) = 0, \pi(2) = 1, \pi(3)=2, \pi(4) = 2, \pi(5)=3, \pi(7,5)=4,
  \dotsb $$
  % Diese Funktion ist eine Treppenfunktion mit Sprungstellen an den Primzahlen.

  Riemannsche Vermutung: $ \Sigma_{n=1}^{\infty} \frac{1}{n^s} = \zeta(s) $

  Sei $p_n$ die $n$-te Primzahl ($p_0 = 2, p_1=3, \dotsb $).
  \subparagraph{Behauptung}
  $ p_n < e^{2^n} $ (Konvention \footnote{$ (a^b)^c = a^{b \cdot c} $, $ a^{b^c}
    =: a^{b^c}$})

  \subparagraph{Beweis per Induktion über $n$}\par
  \textbf{n=0}\par
  $p_0 = 2; e^{2^0} = e^1 = e > 2$\par
  \textbf{$n \implies n+1$}
  $$p_{n+1} \stackrel{\text{Euklid}}{\leq} p_0 \dots p_n + 1 = e^{2^0 + 2^1 +
    \dotsb + 2^n} + 1 = e^{2^{n+1}-1}+1 = e^{2^{n+1}}(\frac{1}{e} +
  \frac{1}{e^{2^{n+1}}}) < e^{2^{n+1}} \;\; \square$$

  \section{Algebraische Strukturen}
  \subsection{Definition: Gruppe}
  Eine Gruppe ist eine Menge $G$ zusammen mit einem ausgezeichneten Element $e
  \in G$ und einer Verknüpfung \\ $ \circ : G \times G \to G, (g,h) \mapsto g \circ
  h$, so dass folgende Axiome gelten:

  \begin{enumerate}
    \item[(G1)] Die Verknüpfung ist assoziativ: $ g \circ ( h \circ k ) = (g \circ
      h) \circ k$ für $g,h,k \in G$
    \item[(G2)] Das Element $e$ ist neutrales Element: $ e \circ g = g = g \circ
      e$ für $g \in G$
    \item[(G3)] Jedes Element besitzt ein Inverses: Für alle $ g\in G$ existiert
      ein $h \in G$ mit $g \circ h = e = h \circ g$
  \end{enumerate}

  Die Gruppe heißt kommutativ (oder \textit{abelsch}), falls zusätzlich gilt:
  \begin{enumerate}
    \item[(G4)] Die Verknüpfung ist kommutativ: $g \circ h = h \circ g$ für alle
      $g,h \in G$.
  \end{enumerate}

  \subsubsection{Beispiele}
  \paragraph{Beispiel}
  $ G = \mathbb{Z}, e = 0 \in \mathbb{Z}, \circ=+ : \Z \times \Z \to \Z $

  \begin{enumerate}
    \item[(G1)] $g+(h+k) = (g+h)+k$ für alle $g,h,k \in \Z$\\ \par
    \item[(G2)] $0+g = g = g+0 $ für alle $g \in \Z$ \\ \par
    \item[(G3)] $g + (-g) = 0 = (-g) + g$ für alle $g \in \Z$ \\ \par
    \item[(G4)] $g+h = h+g$
  \end{enumerate}

  \paragraph{Beispiel} $(\mathbb{Q}, 0, +)$ ist genauso eine abelsche Gruppe.
  \paragraph{Beispiel} $(\Nzero, 0, +)$ ist \textbf{keine Gruppe}.
  \paragraph{Beispiel} $(\Z, 1, \cdot)$ ist \textbf{keine Gruppe}, da G3 nicht
  erfüllt (z.B. existiert kein $n \in \Z mit 2 \cdot n = 1$).
  \paragraph{Beispiel} $(\mathbb{Q}, 1, \cdot)$ ist keine Gruppe, da G3 nicht
  erfüllt (Es existiert kein $x \in \mathbb{Q} \text{mit} 0 \cdot x=1$)
  \paragraph{Beispiel} $(\mathbb{Q*}, 1, \cdot)$, wobei $\mathbb{Q}*:=\mathbb{Q}
  \setminus {0}$ ist eine Gruppe
  \paragraph{Beispiel} $(\Q \setminus \Z, 1, \cdot)$ ist alles, aber keine
  Gruppe

  \subsubsection{Aussage}
  Sei $G$ eine Gruppe mit zwei neutralen Elementen $e, e'$. Dann gilt $e = e'$.

  \paragraph{Beweis}
  $e = e \circ e' = e'$, da $e$ neutral und $e'$ neutral. $\square$

  \paragraph{Bemerkung}
  Analog zeigt sich, dass das Inverse zu einem Element eindeutig bestimmt ist.

  \subsubsection{Aussage}
  Sei $G$ eine Gruppe. Seien $a,b \in G$ mit Inversen $a^{-1} \text{ bzw. }
  b^{-1} \in G$.
  Dann ist $b^{-1} \cdot a^{-1}$ invers zu $(a \circ b)$ \\ $=: ( a \circ
  b)^{-1}$

  \paragraph{Beweis}
  $$ (b^{-1} \circ a^{-1}) \circ (a \circ b) = b^{-1} \circ (a^{-1} \circ a)
\circ b = b^{-1} \circ b = e $$
  \subparagraph{Analog}
  $$ (a \circ b) \circ (b^{-1} \circ a^{-1}) = \dotsb = e $$

  \paragraph{Schreibweise}
%  $\circ$ ist Konvention, manchmal auch als $\cdot$ geschrieben.
  Auch in abstrakten Gruppen schreiben wir häufig $\cdot$ statt $\circ$ für die
  Verknüpfung und $1$ für das neutrale Element.
  \textbf{Abkürzung} $ab := a \cdot b, a^{-1} := \text{Inverses zu a}$.

  \paragraph{Aussage}
  Sei $G$ eine (multiplikativ geschriebene) Gruppe. Für $ a \in G$ gilt dann
  $(a^{-1})^{-1} = a$
  \subparagraph{Beweis}
  $a \cdot a^{-1} = 1 = a^{-1} \cdot a\;\;\;\square$

  \subparagraph{Beispiel}
  [ Gleichseitiges Dreieck mit gegen den Urzeigersinn nummerierten Ecken 1 - 3]\\
  Symmetrien in der Ebene: $ \left\{ \begin{pmatrix} 1 & 2 & 3 \\ 1 & 2 &
      3 \end{pmatrix}, \begin{pmatrix} 1 & 2 & 3 \\ 3 & 1 & 2 \end{pmatrix}, \begin{pmatrix} 1 & 2 & 3 \\ 2 &
      3 & 1 \end{pmatrix} \right\} =: G$ \\
  mit $e$, $\tau$, $\sigma$.\\
  Seien $g,h \in G$. Dann sei $g \cdot h$ die Hintereinanderausführung von $h$
  und danach $g$.

  \subparagraph{Beispiel}
  $$ \begin{pmatrix} 1 & 2 & 3 \\ 3 & 1 & 2 \end{pmatrix} \circ \begin{pmatrix}
    1 & 2 & 3 \\ 2 & 3 & 1 \end{pmatrix} = \begin{pmatrix} 1 & 2 & 3 \\ 1 & 2 &
    3 \end{pmatrix}$$
  $\tau \circ \sigma = e$

  Gruppentafel:\\
  \begin{tabular}{ c | c | c | c }
	  a $\setminus$ b & $e$ & $\sigma$ & $\tau$ \\ \hline
		$e$             & $e$ & $\sigma$ & $\tau$ \\ \hline
		$\sigma$        & $\sigma$ & $\tau$ & $e$ \\ \hline
		$\tau$          & $\tau$ & $e$ & $\sigma$ \\
	\end{tabular}

  \subparagraph{Beispiel}
  [ Gleichseitiges Dreieck mit gegen den Urzeigersinn nummerierten Ecken 1 - 3]\\
  Symmetrien im Raum: $$ \left\{
    \begin{pmatrix} 1 & 2 & 3 \\ 1 & 2 & 3 \end{pmatrix},
    \begin{pmatrix} 1 & 2 & 3 \\ 3 & 1 & 2 \end{pmatrix},
    \begin{pmatrix} 1 & 2 & 3 \\ 2 & 3 & 1 \end{pmatrix},
    \begin{pmatrix} 1 & 2 & 3 \\ 1 & 3 & 2 \end{pmatrix},
    \begin{pmatrix} 1 & 2 & 3 \\ 3 & 2 & 1 \end{pmatrix},
    \begin{pmatrix} 1 & 2 & 3 \\ 2 & 1 & 3 \end{pmatrix}
  \right\}$$
  mit $e$, $\tau$, $\sigma$, $\alpha_1$, $\alpha_2$, $\alpha_3$.

  $\alpha_1 \circ \sigma = \alpha_2$, $\sigma \circ \alpha_1 = \alpha 3 \neq
  \alpha_2 = \alpha1 \circ \sigma$
  Also nicht abelsch / kommutativ.

  $$ \alpha_1^2 = \alpha_1 \circ \alpha_1 = e \implies \alpha_1^{-1} =
  \alpha_1$$

  \paragraph{Definition: symmetrische Gruppe}
  Die \textbf{symmetrische Gruppe in $n$ Buchstaben} ist die Gruppe der
  Permutationen von ${1, \dots, n}$, geschrieben $S_n$, d.h. $S_n =
  \left\{ \begin{pmatrix} 1 & 2 & \cdots & n \\ \sigma_1 & \sigma_2 & \cdots &
      \sigma_n \end{pmatrix} | (\sigma_1, \dotsb, \sigma_n) \text{ Permutationen
      von } (1, \dotsb, n) \right\}$

  \subparagraph{Beispiel}
  $\{\text{Dreiecks-Symmetrie im Raum}\} = S_3$

  \subsubsection{Definition: Untergruppe}
  Eine Teilmenge $U \subseteq G$ einer Gruppe $G$ heißt \textbf{Untergruppe},
  falls (U1) $e \in U$, (U2) $g,h \in U \implies g \circ h \in U$, \\ (U3) $g \in U
  \implies g^{-1} \in U$

  \subparagraph{Beispiel}
  $\{\text{Dreiecks-Symmetrien in der Ebene}\} \subseteq \{Dreiecks-Symmetrien im
  Raum\}$
  \subparagraph{Beispiel}
  $\Z \subseteq (\Q, 0, +)$ ist Untergruppe

  \subparagraph{Beispiel}
  $\Nzero \subseteq (\Z, 0, +)$ ist keine Untergruppe.

  \subsubsection{Definition: Kommutative Ringe}
  Ein \textbf{kommutativer Ring} ist eine Menge $R$ zusammen mit zwei
  ausgezeichneten Elementen $0$ und $1 \in R$ und zwei Verknüpfungen $ + : R \times R \mapsto R$ und $\cdot : R \times R \mapsto R$ so dass gilt:

  \begin{enumerate}
    \item[(R1)] $ \forall x,y,z \in R:   x + (y + z) = (x + y) + z $
    \item[(R2)] $ \forall x \in R: x + 0 = x = 0 + x $
    \item[(R3)] $ \forall x \in R \;\exists\; y \in R : x + y = 0 = y + x $
    \item[(R4)] $ \forall x,y \in R : x + y = y + x $
    \item[(R5)] $ \forall x,y,z \in R : x \cdot (y \cdot z) = (x \cdot y) \cdot
      z $
    \item[(R6)] $ \forall x \in R : x \cdot 1 = x = 1 \cdot x $
    \item[(R7)] $ \forall x,y \in R: x \cdot y = y \cdot x $
    \item[(R8)] $ \forall x,y,z \in R : x \cdot (y + z) = x \cdot y + x \cdot z
      \land ( y + z) \cdot x = y \cdot x + u \cdot x $
  \end{enumerate}

  \paragraph{Beispiel}
  $(\Z, 0, 1, +, \cdot)$

  \paragraph{Beispiel}
  $(\Q, 0, 1, +, \cdot)$


  \paragraph{Beispiel}
  \subparagraph{Menge der Polynome bis $X$ aus $\Z$}
  $ \Z [ X ] = \{ a_nX^n + \dotsb + a_1X+a_0 \;\;|\;\; a_0, \dotsb, a_n \in \Z
  \} $

  \subparagraph{Beispiel}
  $ ( \Z[X], 0, 1, +, \cdot )$

  $ (R[X], 0, 1, +, \cdot) $ falls $R$ kommutativer Ring.

  \subparagraph{Bemerkung}
  Ist $(R, 0, 1, +, \cdot)$ ein kommutativer Ring, so ist $(R, 0, +)$ eine
  abelsche Gruppe.

  \paragraph{Definition}
  Ist $R$ ein kommutativer Ring, so $ R* := \{ x \in R \;| \;\;\exists\; y \in R :
  x \cdot y = 1 = y \cdot x \} $
  Es ist $(R*, 1, \cdot)$ eine kommutative Gruppe, die \textbf{Einheitengruppe
    von $R$}.
  \subparagraph{Beispiel}
  $ \Z* = \{ \pm 1\}, \Q* = \Q \setminus \{ 0 \} $


  \paragraph{Definition: Körper}
  Ein \textbf{Körper} $K$ der Menge ist ein kommutativer Ring für den Multiplikation
  und Addition abelsch definiert sind.
  Somit gelten für ihn die Axiome der abelschen Gruppen $(K,+,0)$ und $(K,
  \cdot, 1)$ und das Distributivgesetz.
  Außerdem ist definiert: $K* = K \setminus \{ 0 \}$

  \subparagraph{Beispiel} $\Q$ und $\mathbb{R}$ sind Körper.

  \subparagraph{Beispiel}
  $ \Q( \sqrt{2} ) := \{ a + b \sqrt{2} \;|\; a,b \in \Q \} \subseteq
  \mathbb{R}$ ist ein Unterkörper.

  $$ 0 = 0 + 0 \sqrt{2} \;,\; 1 = 1 + 0 \sqrt{2} $$
  % $$ (a + b ) $$

  $$ \frac{1}{a+b\sqrt{2}} = \frac{a-b\sqrt{2}}{(a + b\sqrt{2})(a - b
    \sqrt{2})} = \frac{a - b\sqrt{2}}{a^2 - 2b^2} = \frac{a}{a^2 -
    2b^2} -  $$
\end{document}
